\documentclass[12pt,leqno]{amsart}
\pagestyle{plain}
\usepackage{latexsym,amsmath,amssymb}
%\usepackage[notref,notcite]{showkeys} 

\setlength{\oddsidemargin}{1pt}
\setlength{\evensidemargin}{1pt}
\setlength{\marginparwidth}{30pt} % these gain 53pt width
\setlength{\topmargin}{1pt}       % gains 26pt height
\setlength{\headheight}{1pt}      % gains 11pt height
\setlength{\headsep}{1pt}         % gains 24pt height
%\setlength{\footheight}{12 pt} 	  % cannot be changed as number must fit
\setlength{\footskip}{24pt}       % gains 6pt height
\setlength{\textheight}{650pt}    % 528 + 26 + 11 + 24 + 6 + 55 for luck
\setlength{\textwidth}{460pt}     % 360 + 53 + 47 for luck



\def\dsp{\def\baselinestretch{1.35}\large
\normalsize}
%%%%This makes a double spacing. Use this with 11pt style. If you
%%%%want to use this just insert \dsp after the \begin{document}
%%%%The correct baselinestretch for double spacing is 1.37. However
%%%%you can use different parameter.


\def\U{{\mathcal U}}









\begin{document}
\bigskip
\centerline{\bf{Compactness}}
\medskip
\noindent \small{Many of these problems are from a collection made by Behnam Esmayli, who also references a collection created by Cezar Lupu.}
\bigskip
\newline
{\bf Common Lemmas:}
\newline
If $f: X \to Y$ is a continuious function between metric spaces $(X,d)$ and $(Y,\varrho)$ then $f(X)$ is bounded.
\begin{proof}
Assume towards a contradiction that $f(X)$ is not bounded.  Therefore, for a fixed $y\in f(X)$ we have that for every $R$ there is some $f(x)$ such that $\varrho(y, f(x)) > R$.  Therefore we can construct some sequence $(x_n)_{n=1}^\infty$ such that $\varrho(f(x_n), y) > n$.  Since $X$ is compact there must be some $(x_{n_k})_{k=1}^\infty$ with $x_{n_k} \to x_0 \, (\in X)$.  Since $f$ is continuous, we have that $f(x_{n_k}) \to f(x_0) \, (\in Y)$.  Since these sequences are covergent, it must be that $\varrho(f(x_{n_k}) ,y) \to \varrho(f(x_0) , y)$.  However, $\varrho(x_{n_k} , y) > n_k > k$ for all $k$.  This is a contradiction and therefore $f(X)$ is bounded.
\end{proof}

\end{document}