\documentclass[12pt,leqno]{amsart}
\pagestyle{plain}
\usepackage{latexsym,amsmath,amssymb}
%\usepackage[notref,notcite]{showkeys} 

\setlength{\oddsidemargin}{1pt}
\setlength{\evensidemargin}{1pt}
\setlength{\marginparwidth}{30pt} % these gain 53pt width
\setlength{\topmargin}{1pt}       % gains 26pt height
\setlength{\headheight}{1pt}      % gains 11pt height
\setlength{\headsep}{1pt}         % gains 24pt height
%\setlength{\footheight}{12 pt} 	  % cannot be changed as number must fit
\setlength{\footskip}{24pt}       % gains 6pt height
\setlength{\textheight}{650pt}    % 528 + 26 + 11 + 24 + 6 + 55 for luck
\setlength{\textwidth}{460pt}     % 360 + 53 + 47 for luck



\def\dsp{\def\baselinestretch{1.35}\large
\normalsize}
%%%%This makes a double spacing. Use this with 11pt style. If you
%%%%want to use this just insert \dsp after the \begin{document}
%%%%The correct baselinestretch for double spacing is 1.37. However
%%%%you can use different parameter.


\def\U{{\mathcal U}}









\begin{document}
\bigskip
\centerline{\bf{Compactness}}
\medskip
\noindent \small{Many of these problems are from a collection made by Behnam Esmayli, who also references a collection created by Cezar Lupu.}
\bigskip
\newline
{\bf Common Lemmas:}
\newline \newline
{\bf (1)} If $f: X \to Y$ is a continuious function between metric spaces $(X,d)$ and $(Y,\varrho)$ then $f(X)$ is bounded.
\begin{proof}
Assume towards a contradiction that $f(X)$ is not bounded.  Therefore, for a fixed $y\in f(X)$ we have that for every $R$ there is some $f(x)$ such that $\varrho(y, f(x)) > R$.  Therefore we can construct some sequence $(x_n)_{n=1}^\infty$ such that $\varrho(f(x_n), y) > n$.  Since $X$ is compact there must be some $(x_{n_k})_{k=1}^\infty$ with $x_{n_k} \to x_0 \, (\in X)$.  Since $f$ is continuous, we have that $f(x_{n_k}) \to f(x_0) \, (\in Y)$.  Since these sequences are covergent, it must be that $\varrho(f(x_{n_k}) ,y) \to \varrho(f(x_0) , y)$.  However, $\varrho(x_{n_k} , y) > n_k > k$ for all $k$.  This is a contradiction and therefore $f(X)$ is bounded.
\end{proof}
$\newline$
{\bf (2)} If $f: X \to \mathbb{R}$ with $X$ compact, then $f$ must achieve a maximum and minimum value.
\begin{proof}
It suffices to show only $f$ achieving a maximum since the cases are the same.  Applying the first lemma we know that $f(X)$ is a bounded subset of $\mathbb{R}$.  Therefore $\alpha := \sup f(X)$ exists.  Since $\alpha$ is the supremum of the range of $f$ for every $\epsilon$ there must be some $x$ such that $\alpha - \epsilon < f(x) \leq \alpha$.  Letting $\epsilon_n := \frac{1}{n}$ we can construct a sequence $(x_n)_{n=1}^\infty$ with $f(x_n) \to \alpha$.  Also since $X$ is compact there must be some $(x_{n_k})_{k=1}^\infty$ with $x_{n_k} \to x_0 \, (\in X)$.  Since $f(x_n)$ is convergent, its subsequences must also be convergent to the same limit.  And by the continuity of $f$ we have:
$$ f(x_0) \leftarrow f (x_{n_k}) \to \alpha $$
By the uniqueness of limits $f(x_0) = \alpha$ is a maximum. 
\end{proof}

$\newline$
{\bf Problems:} \newline
{\bf (1)} For $f: X \to \mathbb{R}$ with $X$ compact and $\forall x \in X \ \ f(x) > 0$ , then there exists $\delta > 0$ such that for all $x \in X \ \ f(x) \geq \delta$
\begin{proof}
By applying lemma 2 we know that $f$ must achieve a minimum for some $x_0 \in X$ and by hypothesis we know that $f(x_0) > 0$.  Therefore we can pick $\delta := f(x_0)$ and we know that for all $x \in X \ \ f(x) \geq f(x_0) = \delta$.
\end{proof}
$\newline$
{\bf (2)} For a set compact subset $K$ and $F$ a closed subset of some metric space $(X,d)$.  Show that if $K\cap F = \emptyset$ then there exists some $\delta >0$ such that:
$$ \forall p \in K \ \ \forall q \in F \ \ d(p,q) \geq \delta  $$
\begin{proof}
Assume towards a contradiction that for all $\delta > 0$ there exists some $p\in K$ and $q \in F$ such that $d(p,q) < \delta$.  Letting $\delta_n := \frac{1}{n}$ for any $n$ we can construct a sequences $(p_n)_{n=1}^\infty$ in $K$ and $(q_n)_{n=1}^\infty$ in F such that:
$$ d(p_n,q_n) \to 0 $$
Since $p_n$ is a sequence in a compact set there must exist some subsequence $(p_{n_k})_{k=1}^\infty$ that is convergent to some $p_0 \in K$.  I claim that the sequence given by $(q_{n_k})_{k=1}^\infty$ is convergent to $p_0$.
$$ d(q_{n_k}, p_0) \leq d(q_{n_k}, p_{n_k}) + d(p_{n_k}, p_0) $$
and we can fix some $n_0$ where for all $n \geq n_0$:
$$ d(q_{n_k}, p_0) \leq d(q_{n_k}, p_{n_k}) + d(p_{n_k}, p_0) < \frac{\epsilon}{2} + \frac{\epsilon}{2} = \epsilon $$
Since the sequence $q_{n_k}$ is convergent sequence in a closed set, its limit point is necessarily included in the set.  Therefore, $p_0 \in F$.  We can conlude that $\{ p_0 \} \subset F\cap K$ which is a contradiction.
\end{proof}
\noindent I am confused here because I seem to have solved this problem without employing the strategy of {\bf (1)}.  My idea is that you could fix an element of $F$ and define a function the is the distance from the points in $K$ to the fixed element of $F$.  There is necessarily a positive mimimum of this function.  For each element of $F$ there is a fixed minimum value, and we could define another funcion on $F$ that ties these elements to their minimum distanced elements in $F$ and I would claim that this set has a minimum positive value in which case the question would follow but I don't kow how to prove the last step! \newline
$\newline$
{\bf(3)} Let $A$ be a non singular linear map $\mathbb{R}^n \to \mathbb{R}^n$ (an $n \times n$ matrix) show that there exists some $\delta > 0$ such that:
$$ ||A(x)|| = ||Ax|| \geq \delta ||x|| \ \ \forall x \in \mathbb{R}^n$$
\begin{proof}
Note that since $A$ is nonsingular: $||Ax|| = 0 \iff ||x|| = 0$ since the only vector mapped to the 0 vector is the 0 vector.  Therefore we can fix an arbitrary closed n-cell of diameter l denoted $C_o$, and another arbitrary n-cell of Diameter $i$ denoted $C_i$ with $0 < i < l$.  Let $C := C_o \setminus C_i$ in $\mathbb{R}^n$.  Since $C$ is a closed and bounded subset of $\mathbb{R}^n$ it is compact.  Define the function $f: C \to \mathbb{R}^n$ by: 
$$ f(x) := \frac{||Ax||}{||x||} $$
$f$ is a continuous function since the standard norm on $\mathbb{R}^n$ is continuous. The domain of $f$ does not include the 0 vector so it is constantly positive and defined over a compact domain.  Using the results of problem {\bf (1)} there is some $\delta > 0$ such that: 
$$f(x) \geq \delta \iff ||Ax|| \geq \delta||x||$$
Since $C$ can be defined to include any element of $\mathbb{R}^n$ it is true over all of $\mathbb{R}^n$ and the proof is complete.
\end{proof}
Oops this one is wrong!  I think I can modify the solution

\end{document}