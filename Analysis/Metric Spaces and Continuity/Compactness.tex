\documentclass[12pt,leqno]{amsart}
\pagestyle{plain}
\usepackage{latexsym,amsmath,amssymb}
%\usepackage[notref,notcite]{showkeys} 

\setlength{\oddsidemargin}{1pt}
\setlength{\evensidemargin}{1pt}
\setlength{\marginparwidth}{30pt} % these gain 53pt width
\setlength{\topmargin}{1pt}       % gains 26pt height
\setlength{\headheight}{1pt}      % gains 11pt height
\setlength{\headsep}{1pt}         % gains 24pt height
%\setlength{\footheight}{12 pt} 	  % cannot be changed as number must fit
\setlength{\footskip}{24pt}       % gains 6pt height
\setlength{\textheight}{650pt}    % 528 + 26 + 11 + 24 + 6 + 55 for luck
\setlength{\textwidth}{460pt}     % 360 + 53 + 47 for luck



\def\dsp{\def\baselinestretch{1.35}\large
\normalsize}
%%%%This makes a double spacing. Use this with 11pt style. If you
%%%%want to use this just insert \dsp after the \begin{document}
%%%%The correct baselinestretch for double spacing is 1.37. However
%%%%you can use different parameter.


\def\U{{\mathcal U}}









\begin{document}
\bigskip
\centerline{\bf{Compactness}}
\medskip
\noindent \small{Many of these problems are from a collection made by Behnam Esmayli, who also references a collection created by Cezar Lupu.}
\bigskip
\newline
{\bf Common Lemmas:}
\newline \newline
{\bf (1)} If $f: X \to Y$ is a continuious function between metric spaces $(X,d)$ and $(Y,\varrho)$ then $f(X)$ is bounded.
\begin{proof}
Assume towards a contradiction that $f(X)$ is not bounded.  Therefore, for a fixed $y\in f(X)$ we have that for every $R$ there is some $f(x)$ such that $\varrho(y, f(x)) > R$.  Therefore we can construct some sequence $(x_n)_{n=1}^\infty$ such that $\varrho(f(x_n), y) > n$.  Since $X$ is compact there must be some $(x_{n_k})_{k=1}^\infty$ with $x_{n_k} \to x_0 \, (\in X)$.  Since $f$ is continuous, we have that $f(x_{n_k}) \to f(x_0) \, (\in Y)$.  Since these sequences are covergent, it must be that $\varrho(f(x_{n_k}) ,y) \to \varrho(f(x_0) , y)$.  However, $\varrho(x_{n_k} , y) > n_k > k$ for all $k$.  This is a contradiction and therefore $f(X)$ is bounded.
\end{proof}
$\newline$
{\bf (2)} If $f: X \to \mathbb{R}$ with $X$ compact, then $f$ must achieve a maximum and minimum value.
\begin{proof}
It suffices to show only $f$ achieving a maximum since the cases are the same.  Applying the first lemma we know that $f(X)$ is a bounded subset of $\mathbb{R}$.  Therefore $\alpha := \sup f(X)$ exists.  Since $\alpha$ is the supremum of the range of $f$ for every $\epsilon$ there must be some $x$ such that $\alpha - \epsilon < f(x) \leq \alpha$.  Letting $\epsilon_n := \frac{1}{n}$ we can construct a sequence $(x_n)_{n=1}^\infty$ with $f(x_n) \to \alpha$.  Also since $X$ is compact there must be some $(x_{n_k})_{k=1}^\infty$ with $x_{n_k} \to x_0 \, (\in X)$.  Since $f(x_n)$ is convergent, its subsequences must also be convergent to the same limit.  And by the continuity of $f$ we have:
$$ f(x_0) \leftarrow f (x_{n_k}) \to \alpha $$
By the uniqueness of limits $f(x_0) = \alpha$ is a maximum. 
\end{proof}
$\newline$
{\bf (3)} If $f: X \to Y$ is continuous with $X$ compact and $Y$ a metric space then the image of $f$ is compact.
\begin{proof} Take an arbitrary sequence $(f(x_n))_{n=1}^\infty$ in ${\rm Im}f$.  There must exist some convergent subsequence $x_{n_k}$ of $x_n$ with $x_{n_k} \to x_0 (\in X)$.  By the continuity of $f$ we know that $f(x_{n_k}) \to f(x_0)$ which proves the compactness of ${\rm Im}f$.

\end{proof}
$\newline$
{\bf (4)} A sequence $(x_n)_{n=1}^\infty$ being convergent to a point $x_0$ is equivalent to every subsequence $(x_{n_k})$ of $(x_n)$ having a convergent subsequence $x_{n_{k_\ell}} \to x_0$.
\begin{proof}
This first way is easy since if $(x_n)$ is convergent to $x_0$ then its subsequences are as well, and subsequences of subsequences, are, of course, subsequences so they all converge to $x_0$ as well.  To show the other way, I will use contraposition.  Consider the negation of $x_n \to x_0$:
$$ \exists \epsilon >0 \ \ \forall N_0 \ \ \exists n\geq N_0  \ \ d(x_n, x_0) \geq \epsilon$$
We can use the above rule to inductively define a subsequence $(x_{n_k})$ of $(x_n)$ such that for all $k$, $\, d(x_{n_k}, x_0) \geq \epsilon$.  This clearly cannot have a subsequence convergent to $x_0$ since all of its elements are of a minimum distance of $\epsilon$ from $x_0$.  We have shown the negation of the second statement which yields the desired equivalance.
\end{proof}
 
$\newline$
{\bf Problems:} \newline
{\bf (1)} For $f: X \to \mathbb{R}$ with $X$ compact and $\forall x \in X \ \ f(x) > 0$ , then there exists $\delta > 0$ such that for all $x \in X \ \ f(x) \geq \delta$
\begin{proof}
By applying lemma 2 we know that $f$ must achieve a minimum for some $x_0 \in X$ and by hypothesis we know that $f(x_0) > 0$.  Therefore we can pick $\delta := f(x_0)$ and we know that for all $x \in X \ \ f(x) \geq f(x_0) = \delta$.
\end{proof}
$\newline$
{\bf (2)} For a set compact subset $K$ and $F$ a closed subset of some metric space $(X,d)$.  Show that if $K\cap F = \emptyset$ then there exists some $\delta >0$ such that:
$$ \forall p \in K \ \ \forall q \in F \ \ d(p,q) \geq \delta  $$
\begin{proof}
Assume towards a contradiction that for all $\delta > 0$ there exists some $p\in K$ and $q \in F$ such that $d(p,q) < \delta$.  Letting $\delta_n := \frac{1}{n}$ for any $n$ we can construct a sequences $(p_n)_{n=1}^\infty$ in $K$ and $(q_n)_{n=1}^\infty$ in F such that:
$$ d(p_n,q_n) \to 0 $$
Since $p_n$ is a sequence in a compact set there must exist some subsequence $(p_{n_k})_{k=1}^\infty$ that is convergent to some $p_0 \in K$.  I claim that the sequence given by $(q_{n_k})_{k=1}^\infty$ is convergent to $p_0$.
$$ d(q_{n_k}, p_0) \leq d(q_{n_k}, p_{n_k}) + d(p_{n_k}, p_0) $$
and we can fix some $n_0$ where for all $n \geq n_0$:
$$ d(q_{n_k}, p_0) \leq d(q_{n_k}, p_{n_k}) + d(p_{n_k}, p_0) < \frac{\epsilon}{2} + \frac{\epsilon}{2} = \epsilon $$
Since the sequence $q_{n_k}$ is convergent sequence in a closed set, its limit point is necessarily included in the set.  Therefore, $p_0 \in F$.  We can conlude that $\{ p_0 \} \subset F\cap K$ which is a contradiction.
\end{proof}
\noindent I am confused here because I seem to have solved this problem without employing the strategy of {\bf (1)}.  My idea is that you could fix an element of $F$ and define a function the is the distance from the points in $K$ to the fixed element of $F$.  There is necessarily a positive mimimum of this function.  For each element of $F$ there is a fixed minimum value, and we could define another funcion on $F$ that ties these elements to their minimum distanced elements in $F$ and I would claim that this set has a minimum positive value in which case the question would follow but I don't kow how to prove the last step! \newline
$\newline$
{\bf(3)} Let $A$ be a non singular linear map $\mathbb{R}^n \to \mathbb{R}^n$ (an $n \times n$ matrix) show that there exists some $\delta > 0$ such that:
$$ ||A(x)|| = ||Ax|| \geq \delta ||x|| \ \ \forall x \in \mathbb{R}^n$$
\begin{proof}
Note that since $A$ is nonsingular: $||Ax|| = 0 \iff ||x|| = 0$ since the only vector mapped to the 0 vector is the 0 vector.  Therefore we can fix the sphere of radius 1, denoted $S$.  This is compact since it is a closed and bounded susbset of $\mathbb{R}^n$.  Since $||x||$ and $||Ax||$ are nonzero for $x \in S$.  We can then define the function $f: S \to \mathbb{R}^n$ as follows:
$$ f(x) := \frac{||Ax||}{||x||} $$
This function is defined over a compact set and is positive for all $x \in S$.  Therefore, problem {\bf (1)} applies and there must exist some $\delta > 0 $ such that:
$$ f(x) = \frac{||Ax||}{||x||} \geq \delta \iff ||Ax|| \geq \delta||x|| \ \ \ \ (x\in S)$$
It remains to show the above for the rest of $\mathbb{R}^n$.  Since this inequality holds trivially for the zero vector, fix some arbitrary non zero vector, $x$, in $\mathbb{R}^n$.  $x$ can be represented as $t\hat x$ with $\hat x$ a unit vector and $t \in \mathbb{R}$ nonzero.  Since $\hat x$ is a unit vector $\hat x \in S$ and we can write:
$$ ||A(x)|| = ||A(t\hat x)|| = t||A(\hat x)|| \geq t\delta||\hat x|| = \delta||x|| $$
$$ \iff ||Ax|| \geq \delta||x|| $$
Therefore the delta found for $S$ is sufficient for every $x \in \mathbb{R}^n$.
\end{proof}  
$\newline$
{\bf (4)} Let $K$ and $F$ be compact and disjoint subsets of $\mathbb{R}^n$.  Show that there exists open sets $U$ and $V$ that contain $K$ and $F$, respectively, such that $U\cap V = \emptyset$
\begin{proof}
Since $K$ and $F$ are both compact sets we can apply the results of problem {\bf (2)} and find $\delta$ such that forall $k \in K$ and $f \in F$ it will be that $d(k,f) \geq \delta$.  By comactness we can also find finite $\delta/2$ dense sets in both $K$ and $F$.  More specifically, there exists:
$$ \{ k_0, k_1, \dots , k_n\} \subset K  \ \ {\rm and } \ \ \{ f_0, f_1, \dots, f_{\ell} \} \subset F $$
With the properties:
$$ \forall k \in K \ \ \exists i \leq n \ \ d(k, k_i) < \frac{\delta}{2} \ \ {\rm and } \ \ \forall f \in F \ \ \exists j \leq \ell \ \ d(f, f_j) < \frac{\delta}{2}$$ 
For each element of the $\delta/2$ coverings define the open sets:
$$ U_{k_i} := B\left(k_i, \frac{\delta}{2}\right) \ \  U_{f_j} := B\left(f_j, \frac{\delta}{2}\right) $$
Now define the open coverings of $K$ and $F$ as:
$$ U_K := \bigcup_{i=0}^n U_{k_i} \ \ {\rm and } \ \ U_F := \bigcup_{j=0}^{\ell} U_{f_j} $$
Clearly $K$ and $F$ are subsets of $U_K$ and $U_F$, respectively.  And $U_K$ and $U_F$ are both open as they are finite unions of open sets.  We must also have that $U_K \cap U_F = \emptyset$.  Otherwise there would exists some $x$ with $x \in U_K$ and $x \in U_F$ which would imply the existance of $k_i \in K$ and $f_j \in F$ such that:
$$ d(k_i,x) < \frac{\delta}{2} \ \ {\rm and } \ \ d(f_j, x) < \frac{\delta}{2} $$
$$ \Rightarrow d(k_i, f_j) \leq d(k_i, x) + d(x,f_j) < \delta $$
Which is a contradiction with $\delta$ being chosen such that every element in $K$ and $F$ are $\delta$ distance apart. 
\end{proof}
$\newline$ 
The next theme that is described in the collection is that taking limits gives items inside of compact sets!
\newline
{\bf (5)} If $K \subset \mathbb{R}^n$ is compact, then there is a farthest point of $K$ to the origin.
\begin{proof} Begin by defining the set $A$ as follows:
$$ A := \{ x \in \mathbb{R} : \exists k \in K \ \ d({\bf0}, k) = x\} $$
Since $K$ is compact it is bounded, $A$ is also bounded above because of this fact.  Therefore we can define $\alpha := \sup A \,  (\in \mathbb{R})$.  We can also define a sequence $(x_n)_{n=1}^\infty$ in $A$ with $x_n \to \alpha$.  This sequence yields a sequence $(k_n)_{n=1}^\infty$ in $K$ such that $d(k_n, {\bf 0}) \to \alpha$.  Since $k_n$ is a sequence in a compact set there exists a futher subsequence $(k_{n_k})_{k=1}^\infty$ such that $k_{n_k} \to k_0 \, (\in K)$.  Since the distance metric is a continuous function and because of the properties of the sequence that $k_{n_k}$ is a subsequence of we have:
$$ d(k_0, {\bf 0}) \leftarrow d(k_{n_k}, {\bf 0}) \rightarrow \alpha $$ 
By the uniqueness of limits, $d(k_0, {\bf 0}) = \alpha$ and this is a maximal distance from the origin!
\end{proof}
$\newline$
{\bf (6)} For $K \subset \mathbb{R}^n$ a compact set that is disjoint from a closed set $F$, then there is a closest point of $K$ to the set $F$.
\begin{proof}
By applying the result of problem {\bf (2)} we know that there is some $\delta > 0$ such that for all $k \in K$ and $f \in F \ \ d(k,f) \geq \delta $.  We can then define the set:
$$ A := \{ x \in \mathbb{R} : \exists k \in K \ \ \exists f \in F \ \ d(f,k) = x  \} $$
By the above, we know that $A$ is bounded below by some positive $\delta$.  Therefore $\beta := \inf A$ exists and is positive as well.  As before, we can fix sequences $(k_n)_{n=1}^\infty$ and $(f_n)_{n=1}^\infty$ such that $d(k_n,f_n) \to \beta$.  Since $(k_n)$ is a sequence in a compact set it omits a sequence $(k_{n_\ell})_{\ell = 1}^\infty$ which is convergent to some $k_0 \in K$.  We know that $d(k_{n_\ell}, f_{n_\ell}) \to \beta$.  It is also true that $d(k_0, f_{n_\ell}) \to \beta$ since:
$$ d(k_0 , f_{n_\ell}) - \beta \leq d(k_0, k_{n_\ell}) + d(k_{n_\ell}, f_{n_\ell}) - \beta < \frac{\epsilon}{2} + \frac{\epsilon}{2} + \beta - \beta = \epsilon  $$
and:
$$ \beta - d(k_0, f_{n_\ell}) =  \beta  - (k_0, f_{n_\ell}) + d(k_0, k_{n_\ell})  - d(k_0, k_{n_\ell}) $$
$$ = \beta + d(k_0, k_{n_\ell}) - ((k_0, f_{n_\ell}) +d(k_0, k_{n_\ell}) ) \leq \beta + d(k_0, k_{n_\ell}) - d(k_{n_\ell}, f_{n_\ell}) $$
$$ < \beta + \frac{\epsilon}{2}  - (\beta - \frac{\epsilon}{2}) = \epsilon $$
Since $(f_{n_\ell})$ is within $\beta$ of $k_0$ in its limit then for some $N_0$ with $\ell \geq N_0$ the sequence $(f_{n_\ell})_{\ell \geq N_0}^\infty$ must be contained in the closed ball, $B$, of radius $2\beta$ around $k_0$.  Since $B$ is closed and bounded in $\mathbb{R}^n$ then it is compact and therefore $(f_{n_\ell})_{\ell \geq N_0}^\infty$ must have a convergent subsequence $f_{n_{\ell_g}} \to f_0$ but since this is also a sequence in $F$ which is closed $f_0$ must be contained in $F$.  I will now show that $d(k_0, f_0) = \beta$:
$$ d(k_0, f_0) - \beta \leq d(k_0, f_{n_{\ell_g}}) + d(f_{n_{\ell_g}}, f_0) - \beta$$
Since we know that $d(k_0, f_{n_{\ell}}) \to \beta$ we can write:
$$ \leq \beta + \frac{\epsilon}{2} + \frac{\epsilon}{2} - \beta = \epsilon $$
Also:
$$ \beta - d(k_0, f_0) \leq \beta - d(k_0, f_0) - d(f_0, f_{n_{\ell_g}}) + d(f_0,f_{n_{\ell_g}}) $$
$$ = \beta + d(f_0, f_{n_{\ell_g}}) - (d(k_0, f_0) + d(f_0, f_{n_{\ell_g}}) \leq \beta + d(f_0, f_{n_{\ell_g}}) - d(k_0, f_{n_{\ell_g}}) $$
$$ < \beta + \frac{\epsilon}{2} - (\beta - \frac{\epsilon}{2}) = \epsilon $$
$$ |\beta - d(k_0, f_0)| < \epsilon \ \  \ \ \text{       for all } \epsilon >0 $$
$$ \iff \beta = d(k_0, f_0) $$
We have found points in $F$ and $K$ which are at a distance apart equal to the infimum of all the distances between K and F which is positive!
\end{proof} 
$\newline$
{\bf (7)} Let $X$ be a metric space and $f: X \to X$ be a contraction mapping.  Suppose that $K \subset X$ is a compact and nonempty set that satisfies $f(K) = K$.  Prove that $K$ contains one single point.
\begin{proof}
Note that since compact sets are bounded they have a finite diameter which is defined to be the supremum of all of the distances of all of the pairs of points in the set.  For a compact set $C$ this is denoted as $diam \, C$.  Also note that for a given compact set $C$ there exist points in $C$ which acieve this diamater.  Therefore if we assume that $K$ has more than one point there exists $x^*, y^* \in K$ such that $d(x^*, y^*) = diam\, K > 0$.  If we were to consider $f(K)$ we can easily show that this is compact.  
\newline
To do this fix some arbitrary sequence in $(y_n)_{n=1}^\infty$ n $f(K)$.  This induces some $(x_n)_{n=1}^\infty$ in $K$.  This has a convergent subsequence $x_{n_k} \to x_0 \in K$.  Since $f$ is a contraction mapping we know:
$$ f(f(x_0), f(x_{n_k})) \leq k d(x_0, x_{n_k}) \to 0 \ \ (k \in [0,1)) $$
$$ \Rightarrow y_{n_k}  = f(x_{n_k}) \to f(x_0) \in f(K) $$
Since $f(K)$ is also compact we can conclude that there must also be points $x^{**}, y^{**} \in f(K)$ such that $d(x^{**}, y^{**}) = diam \, f(K)$. Also there must exist $x_0, y_0 \in K$ for whice $x^{**} = f(x_0)$ and $y^{**} = f(y_0)$
$$d(x^{**}, y^{**}) =  d(f(x_0), f(y_0)) < d(x_0, y_0) \leq diam \, K $$
$$ \Rightarrow diam \, f(K) < diam \, K \Rightarrow  K \not= f(K) $$
Therefore, $K$ cannot have multiple points!
\end{proof}
$\newline$
{\bf (8)} Let $r: [a,b] \to \mathbb{R}^n$ be a conitinuous curve (the path of a particle).  Suppose that $r(a) = (0, 0, \dots, 0)$ and $r(b) = (1,1, \dots, 1)$.  Prove that there exists a last time the particle touches the surface of the unit sphere. 
\begin{proof}
Since $r$ is continuous with a point inside the unit sphere and a point outside of the unit sphere we know that there exists at least one point where the particle is on the unit sphere.  Therefore we can define the set:
$$ A := \{ x\in \mathbb{R}^n : d(r(x), {\bf 0}) = 1 \} $$
This is nonempty and bounded above and below by $b$, and $a$ respectively!  Therefore the supremum and infimums of $A$ exist and are themselves defined as inputs of $r$.  Let $\alpha := \inf A$ and $\beta := \sup A$.  The notion that $f(\alpha)$ is on the unit sphere is intuitively obvious and not the question so I will only prove that $f(\beta)$ is on the unit sphere.  The proofs are also exactly the same!  \newline
Assume towards a contradiction that $f(\beta)$ is not on the unit sphere $\Rightarrow d(f(\beta), {\bf 0}) \not = 1$.  Therefore we are left with two cases: \newline
[Case $1$] $d(f(\beta), {\bf 0} ) < 1$ $\newline$
Because of the above, there must exist an $\epsilon$ such that $d(f(x), {\bf 0}) < 1 - \epsilon$.  We can then consider the ball $B$ of radius $\epsilon$ centered at $f(\beta)$.  Elements of this ball are also within the unit ball and therefore of distance less than 1 of the origin.  Becauase for $\gamma \in B(f(\beta), \epsilon)$
$$ d({\bf 0}, \gamma) \leq d({\bf 0}, f(\beta)) + d(f(\beta), {\bf 0}) < 1 - \epsilon + \epsilon = 1 $$
By the continuity of $r$ there must exist some $\delta$ where if we consider $\xi$ with $|x - \xi| < \delta$.  And if we let $\delta'$ be the minimum of $\delta$ and $b - \beta$ we can find some $\xi < \beta$ such that $d(f(\xi), {\bf 0}) < 1$.  The existance of $\xi$ contradetics $\beta$ being the supremum of $A$. \newline
[Case 2:] $d(f(\beta), {\bf 0}) > 1$ $\newline$
The argument is exactly the same. \newline
Therefore $\beta$ must be on the unit sphere and is the "last" time the particle is on the unit sphere.  
\end{proof}
$\newline$
The next theme is about every sequence having a convergent subsequence which converges in the set! \newline
{\bf (9)} For $A \subset \mathbb{R}$ with $A$ compact and $x \in A$.  Suppose that every convergent subsequence of $(x_n)_{n\in \mathbb{N}}$ converges to $x$.  Show that the entire sequence converges to $x$.  This was kind of confusing since it did not state that the sequence was in the compact set, but eh whatever.
\begin{proof}
An equivalence to the definition of convergence is that, for a sequence $(x_n)_{n \in \mathbb{N}}$ if there is some $x_0$ where every sequence $(x_{n_k})_{k\in \mathbb{N}}$ has a further subsequence $(x_{n_{k_\ell}})_{\ell \in \mathbb{N}}$ with $x_{n_{k_\ell}} \to x_0$.  Then we can conclude $x_n \to x_0$.  These facts are equivalent, I'll include it as a lemma (4 probably).  To apply this fact to solve this problem fix an arbitrary subsequence $(x_{n_k})$ of $(x_n)$.  Since $(x_{n_k})$ is a sequence in $A$ which is compact there is a convergent subsequence of it $(x_{n_{k_\ell}})$.  Since $(x_{n_{k_\ell}})$ is itself a subsequence of $(x_n)$, by hypothesis we know that $x_{n_{k_\ell}} \to x$.  Since every sequence has a futher subsequence convergent to the same point $(x_n)$ is convergent!
\end{proof} 
$\newline$
{\bf (10)} Let $X$ be a compact metric space and $f: X \to X$ be a contraction map:
$$ \forall x,y \ \ f(f(x), f(y)) < d(x,y) $$
Show that $f$ has a unique fixed point
\begin{proof}
Let the set $\Psi$ be defined as:
$$ \Psi := \{ \gamma \in \mathbb{R} : \exists x\ \ \gamma = d(x , f(x)) \} $$
We know that $A$ is bounded below by $0$ therefore $\alpha := \inf \Psi$ exists.  I will show that there exists a point $x$ such that $d(x, f(x)) = \alpha$.  Because $\alpha$ is an infimum we can find a sequence $(x_n)$ with $d(x_n, f(x_n)) \to \alpha$.  Since $x_n$ is a sequence in $X$, which is compact we have $x_{n_k} \to x_0$ and $d(x_0, f(x_0)) = \alpha $.    With this information we must conclude that $\alpha = 0$.  If $\alpha > 0$ then $x_0 \not= f(x_0)$ and since $x_0,f(x_0) \in X$ we can calculate $f(x_0), f(f(x_0)) \in X$.  By the contraction property of $f$ we know that $d(f(x_0), f(f(x_0))) < d(x_0, f(x_0)) = \alpha$ but this cannot be since $\alpha$ is a lower bound.  We must conclude that $\alpha = 0$ and therefore $d(x_0, f(x_0)) = 0 \iff f(x_0) = x_0$.
\newline
To show uniqueness, assume that two points $x_0, y_0$ satisfy this property.  Construct the set $\{x_0, y_0\}$ which is a compact set.  We should have that $f(\{x_0, y_0\}) = \{x_0, y_0\}$ but this is a contradiction with the results of problem {\bf (7)}.
\end{proof}
$\newline$
The next theme is about continuous functions on compact sets are automatically uniformally continuous!  Ie for $f : X \to Y$ with $X$ compact, then $f$ is uniformally continuous.
\begin{proof}
Assume the negation of uniform continuity:
$$ \exists \epsilon >0 \ \ \forall \delta >0 \ \ \exists x,y \in X \ \  d(x,y) < \delta \ \ {\rm and} \ \ \varrho(f(x), f(y)) \geq \epsilon$$
Using this rule we can then let $\delta_n := \frac{1}{n}$ and find sequences $(x_n)$ and $(y_n)$ with $d(x_n,y_n) \to 0$ and $\varrho(f(x_n), f(y_n)) \geq \epsilon$.  By compactness of $X$ there exists some $(x_{n_k}) \to x_0$ and we have that $d(x_0. y_{n_k}) \to 0$ which means that $y_{n_k} \to x_0$.  However $\varrho(f(x_{n_k}), f(y_{n_k})) \geq \epsilon$ for all $k$ and this is a contradiction.
\end{proof}
$\newline$
{\bf (11)} Let $X$ be a compact subset of $\mathbb{R}^n$, and $f: X \to \mathbb{R}$ be continuous.  Prove that any given $\epsilon > 0$ there exists some $M$ such that for all $x,y$:
$$ |f(x) - f(y)| \leq M|x-y| + \epsilon $$
\begin{proof}
We know that since $f$ is defined over a compact space and is continuous it must be uniformally continuous.  Letting $\epsilon$ be given we can find some $\delta_\epsilon >0$ where for all $x,y$:
$$ d(x,y) < \delta_\epsilon \Rightarrow \varrho(f(x), f(y)) < \epsilon $$
For those $x,y$ with $d(x,y) \geq \delta_\epsilon$ $X$ and ${\rm Im}\, f$ are compact we have the inequalities:
$$ \delta  \leq |x-y| \leq diam\,X \text{ and } \varrho(f(x), f(y)) < diam \, {\rm Im}\,f $$
If we let $M$ be such that $M\delta_\epsilon > diam \, {\rm Im}\, f$ then for arbitrary $x,y$: \newline
If $d(x,y) < \delta_\epsilon$:
$$ \varrho(f(x), f(y)) < \epsilon \leq Md(x,y) + \epsilon $$
If $d(x,y) \geq \delta_\epsilon$:
$$ \varrho(f(x), f(y)) < diam \, {\rm Im}\, f < M\delta_\epsilon \leq Md(x,y) + \epsilon $$
\end{proof}
Slightly confused here, was it meant that $M$ is dependant on $\epsilon$?  Also I did not utilize any of the properties from $\mathbb{R}$ or $\mathbb{R}^n$ and this works for continuous functions defined over any compact spaces.
\newline \newline
{\bf (12)} Suppose that $X$ is a compact metric space, $Y$ is a metric space and $f : X \to Y$ is one-to-one, onto, and continuous.  Show that $f^{-1} : Y \to X$ is continuous.  Is the compactness of $X$ necessary? 
\begin{proof}
Since we know that $f^{-1}$ exists we can fix some arbitrary sequence $(y_n)_{n=1}^\infty$ convergent to $y_0$ in ${\rm Im}\, f$.  $f^{-1}$ is continuous iff:
$$ f^{-1}(y_n) \to f^{-1}(y_0) $$
We know that for all $n$, $y_n = f(x_n)$ and $y_0 = f(x_0)$.  So, $y_n$ omits a sequence $(x_n)$ in $X$.  Furthermore we can equivalenty state the above as:
$$  f^{-1}(y_n) \to f^{-1}(y_0) \iff x_n \to x_0 $$
Suppose that $x_n \not\to x_0$.  This implies the existance of some subsequence $(x_{n_k})$ such that for all $k$, $d(x_{n_k}, x_0) \geq \epsilon$.  There must be a subsequence  $(x_{n_{k_\ell}})$ that is convergent to $x_1 \not= x_0$ because $X$ is compact.  Since $y_n \to y_0 \iff f(x_n) \to f(x_0)$ it must be that $f(x_{n_{k_\ell}}) \to f(x_0)$ but also by the continuity of $f$ we have that $f(x_{n_{k_\ell}}) \to f(x_1)$ which are not the same point by the one to one property of $f$.  This is a contradiction and we habe that:
$$ x_n \to x_0 \iff f^{-1}(y_n) \to f^{-1}(y_0) $$
Which shows the continuity of $f^{-1}$.
\end{proof}
$\newline$
{\bf (12)} This counter example shows why the above requires compactness.  Take the interval $[0,1)$ and wrap it around the unit circle $\mathbb{S}^1$.  Letting $f: [0,1) \to \mathbb{S}^1$ be this continuous mapping with $\lim_{x\to 1^-}f(x) = f(0)$.
\begin{proof}
Such a mapping can be paramterized by the following:
$$ f: [0,1) \to \mathbb{S}^1 \ \ \ \  f(t) :=  (\cos(2\pi t), \sin(2\pi t)) $$
The surjectivity and injectivity properties of $f$ follow from this, as does continuity.  However, if we consider $f^{-1}: \mathbb{S}^1 \to [0,1)$ and in particular at the point $(1,0)$.  For any gap $\delta$ around $(0,1)$ we can find points on $[0,1)$ close the $0$ that map within $\delta$ of $(0,1)$ and points close to $1$ which map to within $\delta$ of $(0,1)$.  If it was required that $[0,1)$ had to be compact then to preserve the injectivity of $f$ the enpoints of the function would have to map to different locations and the above contradiction would not be obtained.
\end{proof}
$\newline$
{\bf (12)} Let $(M,d)$ be a compact metric space and $T: M \to M$ satisfy:
$$ d(x,y) \leq d(T(x), T(y)) $$
for all pairs, ie, distances are non decreasing under T.  Define $x_1 := T(z)$ and $x_{n+1} := T(x_n)$.  Show that some subsequence of $(x_n)$ converges to $z$. 
\end{document}
